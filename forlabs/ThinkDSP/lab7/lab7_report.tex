\documentclass[a4paper,12pt]{article}
\usepackage{cmap}
\usepackage[T2A]{fontenc}
\usepackage[utf8]{inputenc}
\usepackage[english,russian]{babel}
\usepackage{listings}
\usepackage{amsmath}
\usepackage{amsfonts}
\usepackage{float}
\usepackage{csquotes}
\usepackage{graphicx}
\usepackage{hyphenat}
\usepackage{xcolor}
\usepackage{hyperref}
\usepackage{mathtools}
\usepackage{upgreek}


\renewcommand{\theequation}{\thesection.\arabic{equation}}


\author{Шерепа Никита}
\title{ThinkDSP. Лабораторная 7. Дискретное преобразование Фурье.}
\date{\today}

\graphicspath{{res/screenshots}}

\begin{document}%
	
	\maketitle
	
	\newpage \tableofcontents
	\newpage \listoffigures
	\newpage \lstlistoflistings
	
	\newpage
	
	\definecolor{dkgreen}{rgb}{0,0.6,0}
	\definecolor{gray}{rgb}{0.5,0.5,0.5}
	\definecolor{mauve}{rgb}{0.58,0,0.82}
	
	\lstset{
		language=Python,                 % выбор ЯП для подсветки 
		basicstyle=\small\sffamily, % размер и начертание шрифта для подсветки кода
		numbers=left,               % где поставить нумерацию строк (слева\справа)
		numberstyle=\tiny,           % размер шрифта для номеров строк
		stepnumber=1,                   % размер шага между двумя номерами строк
		numbersep=5pt,                % как далеко отстоят номера строк от подсвечиваемого кода
		aboveskip=3mm,
		belowskip=3mm,
		showstringspaces=false,
		columns=flexible,
		captionpos=b, 
		basicstyle={\small\ttfamily},
		numbers=left,
		numberstyle=\tiny\color{gray},
		keywordstyle=\color{blue},
		commentstyle=\color{mauve},
		stringstyle=\color{dkgreen},
		breaklines=true,
		breakatwhitespace=true,
		tabsize=3
	}

	\section{Упражнение 7.2}
	
	\begin{enumerate}
		
		\item{Задание}
		
		Дан массив сигнала \texttt{y}. Разделите его на четные элементы \texttt{e} и нечетные элементы \texttt{o}.
		
		Вычислите ДПФ \texttt{e} и \texttt{о}, делая рекурсивные вызовы
		
		Вычислите ДПФ(\texttt{y}) для каждого значения \texttt{n}, используя лемму Дэниелсона-Ланцоша.
		
		\item{Ход работы}
		
		
		Рассмотрим сигнал и вычислим его БПФ
		
		\begin{lstlisting}[caption=БПФ сигнала]
			ys = [-0.5, 0.1, 0.7, -0.1]
			hs = np.fft.fft(ys)
			print(hs)
			
			Output
			[ 0.2+0.j  -1.2-0.2j  0.2+0.j  -1.2+0.2j]
		\end{lstlisting}
	
		Теперь реализуем ДПФ
		\begin{lstlisting}[caption=ДПФ]
			def dft(ys):
			N = len(ys)
			ts = np.arange(N) / N
			freqs = np.arange(N)
			args = np.outer(ts, freqs)
			M = np.exp(1j * PI2 * args)
			amps = M.conj().transpose().dot(ys)
			return amps
		\end{lstlisting}
		
		Применим ДПФ к нашему сигналу
		\begin{lstlisting}[caption=Применение ДПФ]
			hs2 = dft(ys)
			np.sum(np.abs(hs - hs2))
			
			Output
			5.864775846765962e-16
		\end{lstlisting}
	
		Видим, что ДПФ даёт почти такой же результат. Разница в результатах = 5.864775846765962e-16
		
		Прежде чем перейти к созданию рекурсивного БПФ, рассмотрим версию, которая использует \texttt{np.fft.fft} для вычисления БПФ половин
		\begin{lstlisting}[caption=Вычисление БПФ половин]
			def fft_norec(ys):
			N = len(ys)
			He = np.fft.fft(ys[::2])
			Ho = np.fft.fft(ys[1::2])
			
			ns = np.arange(N)
			W = np.exp(-1j * PI2 * ns / N)
			
			return np.tile(He, 2) + W * np.tile(Ho, 2)
		\end{lstlisting}
		
		Применим её к сигналу
		\begin{lstlisting}[caption=Применение к сигналу]
			hs3 = fft_norec(ys)
			np.sum(np.abs(hs - hs3))
			
			Output
			0.0
		\end{lstlisting}
		
		Видим, что результат такой же. Разница = 0.0
		
		Теперь заменим \texttt{np.fft.fft} на рекурсивные вызовы
		\begin{lstlisting}[caption=Добавление рекурсивных вызовов]
			def fft(ys):
			N = len(ys)
			if N == 1:
			return ys
			
			He = fft(ys[::2])
			Ho = fft(ys[1::2])
			
			ns = np.arange(N)
			W = np.exp(-1j * PI2 * ns / N)
			
			return np.tile(He, 2) + W * np.tile(Ho, 2)
		\end{lstlisting}
		
		И применим результат к сигналу
		\begin{lstlisting}[caption=Применение к сигналу]
			hs4 = fft(ys)
			np.sum(np.abs(hs - hs4))
			
			Output
			1.6653345369377348e-16
		\end{lstlisting}
		
		Разница в результатах = 1.6653345369377348e-16
		
		Данная реализация имеет сложность O(nlogn) по времени и по памяти.
		
		
	\end{enumerate}
	
	
	\section{Вывод}
	
	Во результате выполнения работы получены навыки работы с Дискретным Преобразованием Фурье (ДПФ) и Быстрым Преобразованием Фурье (БПФ).
	
	
\end{document}